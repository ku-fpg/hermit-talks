%include frontMatter-gill.lhs.tex

%%%%%%%%%%%%%%%%%%%%%%%%%%%%%%%%%%%

\title[HERMIT]{Making a Century in HERMIT}
%\subtitle{A Plugin for the Interactive Transformation\\ of GHC Core Language Programs}

\author[Andrew Gill]{Neil Sculthorpe, Andrew Farmer, and Andrew Gill}


\institute{Functional Programming Group\\
           Information and Telecommunication Technology Center\\
           University of Kansas}

\date{2nd October 2014}

\renewcommand{\section}[1]{}

\begin{document}

%%%%%%%%%%%%%%%%%%%%%%%%%%%%%%%%%%%

\begin{frame}
  \titlepage
\end{frame}

%%%%%%%%%%%%%%%%%%%%%%%%%%%%%%%%%%%

%-include fullemployment.tex
%include spam.tex
%include motivation.lhs.tex
%include demo-fib.lhs.tex
\begin{frame}[verbatim]

{\Huge First Demo}

\pause
\begin{itemize}
\item {resume}                    \dotfill{} resume the compile
\item {\tt binding-of 'main}    \dotfill{} goto the {\tt main} definition
\item {\tt binding-of 'fib} \dotfill{} goto the {\tt fib} definition
\item {\tt remember "myfib"}    \dotfill{} remember a definition
\item {\tt show-remembered}     \dotfill{} show what has been remembered
\item {any-call (unfold-remembered "myfib")} \dotfill{} try unfold ``myfib''
\item {\tt bash}                \dotfill{} bash a syntax tree with simple rewrites
\item {\tt top}                 \dotfill{} go back to the top of the syntax tree
\item {\tt load-and-run "Fib.hss"} \dotfill{} load and run a script
\end{itemize}

\end{frame}

%include download.lhs.tex
%include forTheSakeOfWhat.tex
%include twoLevel.tex
%include hermitProject.lhs.tex
%include commands.lhs.tex

%%include workerWrapper.tex
%%include ww-demo.tex


\begin{frame}

\frametitle{Pearls of Functional Algorithm Design}

\begin{centering}

\includegraphics[width=1in]{Pictures/bird.pdf}

\end{centering}

\begin{itemize}
\item The book is entirely dedicated to reasoning about Haskell programs, with each chapter calculating an efficient program from an inefficient specification program.
\item We selected the chapter \emph{Making a Century} from the textbook \emph{Pearls of Functional Algorithm Design}.
\end{itemize}


\end{frame}

\begin{frame}
\frametitle{Larger Example: Deriving a better century}

The program in \emph{Making a Century} computes the list of all arithmetic expressions formed from ascending digits, where juxtaposition, addition, and multiplication evaluate to 100.
For example, one possible solution is
\[
100 = 12 + 34 + 5 \times 6 + 7 + 8 + 9
\]

The derivation of an efficient program involves a substantial amount of equational reasoning, and the use of a variety of proof
techniques, including fold/unfold transformation, structural induction, fold fusion, and numerous auxiliary lemmas.

\end{frame}

\begin{frame}[verbatim]

Fragment of the proof:

\tiny
\begin{verbatim}
{ rhs-of 'solutions

  -- filter (good . value) . expressions

        { app-arg ; inline 'expressions }

  -- filter (good . value) . foldr extend []

        { [ app-fun, app-arg ] ; apply-rule "6.2" }
        { lemma "comp-assoc" }

  -- filter (good . value) . filter (ok . value) . foldr extend []

        { app-arg

          -- filter (ok . value) . foldr extend []

              { lemma "foldr-fusion-1" }

          -- foldr extend' []

        }

  -- filter (good . value) . foldr extend' []

}
\end{verbatim}


\end{frame}


%include rules.lhs.tex

\begin{frame}[verbatim]
\frametitle{What happened while deriving a better century}

\begin{itemize}
\item  During mechanization we discovered that several auxiliary properties in the textbook
  are stated as assumptions without proof.
\begin{itemize}
\item we suspect that they are deemed either ``obvious'' or ``uninteresting''.
\end{itemize}
\item Assumption 6.2 also had a simple proof, but it relied on arithmetic properties of Haskell's built-in \verb|Int| type
(specifically, that \verb|m == n| $\implies$ \verb|m <= n|).
\item Two proof techniques are used in the textbook that HERMIT does not directly support.
\begin{itemize}
\item The first is the fold fusion law, which needs implication, which we do not support.
\item The second involves postulating the existence of an auxiliary function.
\item We did manage to run the postulated function backwards, to verify the calculation.
\end{itemize}
\item We have a plugin that provides the fold fusion law as a primitive.
\end{itemize}
\end{frame}


\begin{frame}
\frametitle{Length of Calculations for Century}

\renewcommand{\tabcolsep}{1ex}

\begin{center}
\begin{tabular}{||l||c||c||c||c||c||}
\hline
\multicolumn{1}{||c||}{\multirow{2}{*}{Calculation}}
             &  Textbook        &  \multicolumn{3}{c||}{HERMIT Commands}\\
\cline{3-5}
             &  Lines           &  Transformation  &  Navigation  & Total \\
\hline
|solutions|  &  16              &  12              & ~7            & ~19 \\
|expand|     &  19              &  18              & 20           & ~38 \\
Lemma 6.5    &  not given       &  ~4               & ~4            & ~~8 \\
Lemma 6.6    &  not given       &  ~2               & ~1            & ~~3 \\
Lemma 6.7    &  not given       &  ~2               & ~0            & ~~2 \\
Lemma 6.8    &  ~7               &  ~5               & ~8            & ~13 \\
Lemma 6.9    &  ~1               &  ~4               & ~4            & ~~8 \\
Lemma 6.10   &  not given       &  23              & 13           & ~36 \\
Total        &  43              &  70              & 57           & 127 \\
\hline
\end{tabular}
\end{center}


\end{frame}

\begin{frame}
\frametitle{HERMIT Summary}

\begin{itemize}

\item
A GHC plugin for interactive transformation of GHC Core programs

\item
Can express basic equational reasoning as HERMIT scripts

\item
New and powerful commands can be defined using a HERMIT plugin mechanism

\end{itemize}

\center{\tt\Large cabal install hermit}

\end{frame}

%%%%%%%%%%%%%%%%%%%%%%%%%%%%%%%%%%%

\end{document}
